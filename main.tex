\documentclass[10pt,a4paper]{article}
\usepackage[utf8]{inputenc}
\usepackage[danish]{babel}
\usepackage{amsmath}
\usepackage{amsfonts}
\usepackage{amssymb}


\author{Rasmus Tollund}
\title{P0 - Selvevaluering}
\date{8. oktober 2018}

\begin{document}
\maketitle

\section*{Formål}
Dette er en selvevaluering af Rasmus Tollund for projektet "P0 - Hvis programmer er løsningen, hvad er så problemet?", som blev udarbejdet i september 2018. Formålet med denne evaluering er at beskrive de erfaringer, som jeg har gjort mig igennem forløbet, og beskrive hvilke ting jeg gjorde godt og skidt. Dette gøres som led i at kunne dokumentere min udvikling til PV eksamen.

\section*{Samarbejde}
Jeg synes selv jeg er ret god til at samarbejde, og er egentlig også ret vant til det. Jeg kan diskutere emnet med gruppemedlemmerne, og er god til at få problemstillinger til diskussion. Jeg kan godt lide processen med, at man sammen kan lave noget der opnår højere niveau end den enkelte kunne selv.

En af mine svagheder er, at jeg nemt bliver irriteret over at andre ikke deltager ligeså aktivt som mig selv, og samtidig har jeg heller ikke lyst til at skulle være sur, og på den måde ødelægge gruppedynamikken. Derfor ender det ofte med, at dem der ikke gider lave så meget, de bare får lov til det, hvilket er meget frustrerende.

De ikke så aktive gruppemedlemmer var nok det eneste konflikt der var. Dette valgte jeg dog ikke at håndtere, hvilket nok var en dum idé. Men det var svært at konfrontere folk, når de var de eneste man kendte, ville jeg helst ikke gøre humøret surt mellem os.

\section*{Projektstyring}
Da vi i P0 ikke har gjort brug af nogen værktøjer til at styreprojektet, kan dette ikke evalueres. Dog havde vi uddelt opgaverne på tavlen, sådan man kunne se hvem der lavede hvad. Hele processen her var dog dårlig, hvilket kan læses i procesevalueringen. Jeg var ikke god nok til, for mig selv og for min gruppe, at strukturere projektet, sådan at jeg kunne finde ud af hvor langt vi var, og hvad der skulle laves. Desuden var jeg for dårlig til, at komme i gang med det der skulle laves i god tid, fordi projektet var ret kedeligt.

\section*{Kurser}
I kursene brugte vi i gruppen tiden efter forelæsningen på at lave opgaver sammen. Dog var de andre gruppemedlemmer ikke helt så godt med, som jeg var. Hvilket gjorde at jeg hurtigt kom foran og skulle lære de andre. Dette er der bestemt ikke noget galt ved, da man lærer meget af at skulle lære fra sig. Det virkede dog ikke til, at de andre i gruppen var særligt trygge ved at spørge om og modtage hjælp. Det var tit svært at afgøre om de havde forstået det, eller om de bare sagde "jaja" for at komme videre.

Starten af kurserne var, for mig, meget nemme, da jeg vidste det meste af imperativ programmering, DTG og PV i forvejen. Det betyder også, at jeg ikke har det store at evaluere omkring mig selv i forhold til læringen.

\section*{P1}
I P1 vil jeg forsøge, at snakke med mine gruppemedlemmer mere om hvad vi forventer af hinanden. Ydermere vil jeg forsøge, at i gruppen sætte milestones, der gør det nemmere at sprede arbejdsbyrden ud over hele forløbet.

Jeg håber, at jeg i P1 projektet vil være mere motiveret, da emnerne virker meget mere spændende. I P0 fik vi gjort det kedeligt for os selv tror jeg, ved at snævre det ind på et problem der var meget lille.


\end{document}
